\documentclass[12pt, a4paper, oneside, romanian]{teza-upb}
\setcounter{secnumdepth}{3}
\setcounter{tocdepth}{3}
\usepackage{babel}
\usepackage{color}
\usepackage{graphicx}
\usepackage{float}
\graphicspath{ {./imagini/} }
\usepackage{indentfirst}
\usepackage[
  bookmarks,
  bookmarksopen=true,
  pdftitle={licenta},
  linktocpage]{hyperref}


\singlespacing

%%%%%%%%%%%%%%%%%%%%%%%%%%%%%%%%%%%%%%%%%%%%%%%%%%%%%%%%cod colorat
\usepackage{listings}
\usepackage{color}
 
\definecolor{codegreen}{rgb}{0,0.6,0}
\definecolor{codegray}{rgb}{0.5,0.5,0.5}
\definecolor{codepurple}{rgb}{0.58,0,0.82}
\definecolor{backcolour}{rgb}{0.95,0.95,0.92}

\lstdefinelanguage{JavaScript}{
  keywords={typeof, new, true, false, catch, function, return, null, catch, switch, var, if, in, while, do, else, case, break},
  ndkeywords={class, export, boolean, throw, implements, import, this},
  sensitive=false,
  comment=[l]{//},
  morecomment=[s]{/*}{*/},
  morestring=[b]',
  morestring=[b]"
}

\lstset{language=JavaScript}

\lstdefinestyle{mystyle}{
	language=JavaScript,
    commentstyle=\color{codegreen},
    keywordstyle=\color{blue},
  	ndkeywordstyle=\color{blue},
    numberstyle=\tiny\color{codegray},
  	identifierstyle=\color{black},
    stringstyle=\color{codepurple},
	basicstyle=\ttfamily\scriptsize,
    breakatwhitespace=false,         
    breaklines=true,                 
    captionpos=b,                    
    keepspaces=true,                 
    numbers=left,                    
    numbersep=5pt,                  
    showspaces=false,                
    showstringspaces=false,
    showtabs=false,                  
    tabsize=2
}
 
\lstset{style=mystyle}
%%%%%%%%%%%%%%%%%%%%%%%%%%%%%%%%%%%%%%%%%%%%%%%

\begin{document}

\author{Albert-Lucian LUTA}

\title{Includerea instrumentelor de predare-testare într-o aplicație e-learning}


\facultatea{Facultatea de Electronică, Telecomunicații și Tehnologia Informației}
\tiplucrare{diploma}
\domeniu{Calculatoare și Tehnologia Informației}
\catedra{Telecomunicații}
\campus{Leu} 
\program{Ingineria Informației}
\titlulobtinut{Inginer}
\director{Ş.L.Dr.Ing. Elena Cristina STOICA} 

\submissionmonth{Septembrie} 
\submissionyear{2021} 

\beforepreface
\listoffigures
\listoftables
\abbreviations{
	% TODO: adauga traducere pentru tot ce este in engleza
	SPA = Single Page Aplication

	CSR = Client Side Rendering

	SSR = Server Side Rendering

	SSG = Static Site Generation

	ISR = Incremental Static Regeneration

	HMR = Hot Module Replacement

	API = Aplication Programming Interface

	SEO = Search Engine Optimization

	FCP = First Contentful Paint

	TCP = Transmission Control Protocol

	UI = User Interface

	UX = User Experience

	ORM = Object-relational Mapping

	SGBD = Sistem de gestiune a bazei de date
}
\afterpreface 

\chapter*{Introducere}
\addcontentsline{toc}{chapter}{Introducere}

\section{Scopul Proiectului}

Proiectul are ca scop dezvoltarea unei aplicatii web e-learning, cu ajutorul careia universitatile sa-si poata desfasura activitatea de predare/testare si in mediu online. Aceasta metoda nu are rolul de a inlocui desfasurarea activitii in mediu fizic, ci de a o imbunatati. Spre exemplu, studentii pot avea acces de oriune si oricand la materialele si notele fiecarei materii, iar profesorii pot primi temele si proiectele pentru curs, seminar sau laborator online, fara a  mai fi nevoie sa transporte foile de hartie ale fiecarui student. Cu toate acestea, mediul online prezinta dezavantaje in momentul testarii/examinarii cunostintelor proprii ale studentilor, deoarece comunicarea online este foarte greu sau chiar imposibil de oprit. Din acest motiv, consider ca examinarile fizice sunt cea mai buna modalitate in aces caz, iar cele online pot fi pastrate pentru simulari, exercitii propuse sau variante din ani trecuti ai examenului.

\section{Descrierea proiectului}

Aplicatia permite crearea si managementul unei universitati. In cadrul universitatii pot exista 3 tipuri de utilizatori: utilizator cu rol de administrator, utilizator cu rol de profesor si utilizator cu rol de student, fiecare avand acces la functionalitati diferite.
\begin{itemize}
	\item Student
		\begin{itemize}
			\item Accesarea materialelor(cursuri, seminarii, laboratoare)
			\item Accesarea temelor si incarcarea rezolvarilor
			\item Participarea la teste si simulari
			\item Participarea la discutii pe forum
			\item Vizualizarea notelor
			\item Vizualizarea viitoarelor activitati
		\end{itemize}
	\item Profesor
		\begin{itemize}
			\item Incarcarea materialelor
			\item Crearea temelor
			\item Crearea testelor/simularilor
			\item Notarea studentilor
			\item Participarea la discutii pe forum
			\item Impartirea activitatilor(materiale, teme, teste, forum) pe sectiuni
			\item Crearea setului personal de intrebari si raspunsuri pentru teste
		\end{itemize}
	\item Administrator
		\begin{itemize}
			\item Managementul facultatilor
			\item Managementul materiilor
			\item Managementul utilizatorilor
		\end{itemize}
\end{itemize}

\chapter{Proiectare}

Inainte de a trece la implementarea propriu-zisa a proiectului, trebuie sa intelegem foarte bine cerintele si metodele prin care acesta va fi construit. Aceasta treapta preliminara implementarii se numeste proiectare, si a devenit din ce in ce mai importanta cu cat programele software au devenit mai complexe.

Proiectarea software reprezinta procesul de gasire a solutiilor conceptuale pentru realizarea unei aplicatii, pe baza unui set de cerinte si asteptari. Acest pas poate reduce costurile si timpul de implementare, in cazul unor neintelegeri initiale sau a unor schimbari de cerinte, si usureaza procesul de estimare al timpului si costului de realizare al intregului proiect.

\begin{figure}[H]
\centering
\includegraphics*[width=0.55\columnwidth]{proiectare-software-model-cascada}
\caption{Proiectare Software - Model Cascada\cite{proiectaresoftwaremodelcascada}}
\label{proiectare-software-model-cascada}
\end{figure}

\section{Functionalitati administrator}

Functionalitatile utilizatorilor cu rol de administrator sunt urmatoarele:
\begin{itemize}
	\item Managementul facultatilor
	\item Managementul materiilor
	\item Managementul utilizatorilor
\end{itemize}

\subsection{Structura universitatii}

Inainte de a putea face managementul utilizatorilor, administratorul este nevoit sa defineasca structura universitatii - facultatile si materiile componente. Acesta va putea adauga facultati, iar pentru fiecare dintre acestea va putea adauga materiile asociate.

\begin{figure}[H]
\centering
\includegraphics*[width=0.5\columnwidth]{diagrama-use-case-creeaza-structura-universitatii}
\caption{Diagrama Use-Case - Creeaza Structura Universitatii}
\label{diagrama-use-case-creeaza-structura-universitatii}
\end{figure}


\begin{figure}[H]
\centering
\includegraphics*[width=0.5\columnwidth]{diagrama-activitate-creeaza-structura-universitatii}
\caption{Diagrama Activitate - Creeaza Structura Universitatii}
\label{diagrama-activitate-creeaza-structura-universitatii}
\end{figure}

\subsection{Managementul utilizatorilor}

Odata stabilita structura universitatii, administratorul va putea adauga utilizatori, le va putea atribui un anumit rol si ii va putea repartiza la facultatile si materiile la care vor avea acces.

\begin{figure}[H]
\centering
\includegraphics*[width=0.525\columnwidth]{diagrama-use-case-adauga-utilizator}
\caption{Diagrama Use-Case - Adauga Utilizator}
\label{diagrama-use-case-adauga-utilizator}
\end{figure}


\begin{figure}[H]
\centering
\includegraphics*[width=0.6\columnwidth]{diagrama-activitate-adauga-utilizator}
\caption{Diagrama Activitate - Adauga Utilizator}
\label{diagrama-activitate-adauga-utilizator}
\end{figure}

\section{Functionalitati profesor}

Functionalitatile utilizatorilor cu rol de profesor sunt urmatoarele:
\begin{itemize}
	\item Incarcarea materialelor
	\item Crearea temelor
	\item Crearea testelor/simularilor
	\item Notarea studentilor
	\item Impartirea activitatilor(materiale, teme, teste, forum) pe sectiuni
	\item Crearea setului personal de intrebari si raspunsuri pentru teste
\end{itemize}

\subsection{Crearea activitatilor}

Activitatile pot fi de 4 tipuri: materiale, tema, test si forum. Pentru crearea unei activitati vom avea nevoie atat de detaliile de baza, pe care le vom gasi la orice tip de activitate, cat si de detaliile specifice fiecarui tip in parte. Pentru o mai buna repartizare logica a activitatilor, acestea vor fi impartite pe sectiuni.

\begin{figure}[H]
\centering
\includegraphics*[width=0.9\columnwidth]{diagrama-use-case-creeaza-activitate}
\caption{Diagrama Use-Case - Creeaza Activitate}
\label{diagrama-use-case-creeaza-activitate}
\end{figure}


\begin{figure}[H]
\centering
\includegraphics*[width=0.65\columnwidth]{diagrama-activitate-creeaza-activitate}
\caption{Diagrama Activitate - Creeaza Activitate}
\label{diagrama-activitate-creeaza-activitate}
\end{figure}

\subsection{Notarea studentilor}

Pentru notare, profesorul este nevoit in primul rand sa selecteze activitatea pentru care se va desfasura procesul. Odata aleasa activitatea, acesta va putea alege unul dintre studentii care au participat la aceasta si ii va putea nota.

\begin{figure}[H]
\centering
\includegraphics*[width=0.65\columnwidth]{diagrama-use-case-noteaza-student}
\caption{Diagrama Use-Case - Noteaza Student}
\label{diagrama-use-case-noteaza-student}
\end{figure}


\begin{figure}[H]
\centering
\includegraphics*[width=0.45\columnwidth]{diagrama-activitate-noteaza-student}
\caption{Diagrama Activitate - Noteaza Student}
\label{diagrama-activitate-noteaza-student}
\end{figure}

\subsection{Crearea setului personal de intrebari}

Pentru a face mai facila refolosirea intrebarilor de la un an la altul, profesorii isi vor putea crea un set personal de intrebari pe care il va folosi la generarea testelor. Acesta va putea crea intrebari de diferite tipuri, repartizate in functie de o categorie, adauga raspunsuri si marca pe cele corecte si gresite.

\begin{figure}[H]
\centering
\includegraphics*[width=0.3\columnwidth]{diagrama-use-case-creeaza-set-personal-de-intrebari}
\caption{Diagrama Use-Case - Creeaza Set Personal de Intrebari}
\label{diagrama-use-case-creeaza-set-personal-de-intrebari}
\end{figure}


\begin{figure}[H]
\centering
\includegraphics*[width=0.75\columnwidth]{diagrama-activitate-creeaza-set-personal-de-intrebari}
\caption{Diagrama Activitate - Creeaza Set Personal de Intrebari}
\label{diagrama-activitate-creeaza-set-personal-de-intrebari}
\end{figure}

\section{Functionalitati student}

Functionalitatile utilizatorilor cu rol de student sunt urmatoarele:
\begin{itemize}
	\item Accesarea materialelor(cursuri, seminarii, laboratoare)
	\item Accesarea temelor si incarcarea rezolvarilor
	\item Participarea la teste si simulari
	\item Vizualizarea notelor
\end{itemize}

\subsection{Accesarea detaliilor fiecarei activitati}

Dupa ce profesorul a creat structura materiei, sectiuni si activitati, studentii vor avea acces la acestea. Ei isi vor putea alege activitatea la care vor sa ia parte, si vizualiza detaliile despre aceasta.

\begin{figure}[H]
\centering
\includegraphics*[width=\columnwidth]{diagrama-use-case-acceseaza-activitate}
\caption{Diagrama Use-Case - Acceseaza Activitate}
\label{diagrama-use-case-acceseaza-activitate}
\end{figure}


\begin{figure}[H]
\centering
\includegraphics*[width=0.6\columnwidth]{diagrama-activitate-acceseaza-activitate}
\caption{Diagrama Activitate - Acceseaza Activitate}
\label{diagrama-activitate-acceseaza-activitate}
\end{figure}

\subsection{Incarcarea rezolvarilor pentru teme}

Pentru activitatile de tip tema, acestia isi vor putea incarca rezolvarile si vizualiza nota, dupa ce tema a fost corectata si notata de catre profesor.

\begin{figure}[H]
\centering
\includegraphics*[width=0.5\columnwidth]{diagrama-use-case-incarca-rezolvare-tema}
\caption{Diagrama Use-Case - Incarca Rezolvare Tema}
\label{diagrama-use-case-incarca-rezolvare-tema}
\end{figure}


\begin{figure}[H]
\centering
\includegraphics*[width=0.55\columnwidth]{diagrama-activitate-incarca-rezolvare-tema}
\caption{Diagrama Activitate - Incarca Rezolvare Tema}
\label{diagrama-activitate-incarca-rezolvare-tema}
\end{figure}


\subsection{Participarea la teste}

Pentru activitatile de tip test, studentii vor putea interactiona in diferite moduri, in functie de data curenta si detaliile testului. Vor avea posibilitatea de a incepe o incercare sau a o continua, daca testul este activ si verifica rezultatele intrebarilor, daca testul a expirat. 

\begin{figure}[H]
\centering
\includegraphics*[width=0.75\columnwidth]{diagrama-use-case-participa-la-test}
\caption{Diagrama Use-Case - Participa la Test}
\label{diagrama-use-case-participa-la-test}
\end{figure}


\begin{figure}[H]
\centering
\includegraphics*[width=0.75\columnwidth]{diagrama-activitate-participa-la-test}
\caption{Diagrama Activitate - Participa la Test}
\label{diagrama-activitate-participa-la-test}
\end{figure}

\section{Functionalitati comune}

Functionalitatile comune sunt urmatoarele:
\begin{itemize}
	\item Participarea la discutii pe forum
\end{itemize}

\subsection{Participarea la discutii pe forum}

Indiferent de rolul pe care un utilizator il are in universitate, acesta va putea participa la discutiile pe forum. Cu toate acestea, doar administratorii sau profesorii pot crea discutii noi.

\begin{figure}[H]
\centering
\includegraphics*[width=0.45\columnwidth]{diagrama-use-case-participa-la-forum}
\caption{Diagrama Use-Case - Participa la Forum}
\label{diagrama-use-case-participa-la-forum}
\end{figure}

\begin{figure}[H]
\centering
\includegraphics*[width=0.5\columnwidth]{diagrama-activitate-participa-la-forum}
\caption{Diagrama Activitate - Participa la Forum}
\label{diagrama-activitate-participa-la-forum}
\end{figure}

\chapter{Implementare}

Fiind o aplicatie web, am folosit arhitectura standard de construire a acestora "client-server". Aceasta imparte aplicatia in doua parti, fiecare avand un rol separat, comunicand intre ele prin conexiuni TCP. Partea de client se refera la interfata vizuala cu care interactioneaza un utilizator in momentul folosirii, iar partea de server se refera la codul care ruleaza pe un calculator, de obicei in cloud\footnote{Ansamblu distribuit de servicii de calcul}, care raspunde cererilor primite de la client, cel din urma stocand datele necesare rularii aplicatiei, de cele mai multe ori intr-o baza de date.

\begin{figure}[H]
\centering
\includegraphics*[width=0.7\columnwidth]{arhitectura-client-server}
\caption{Arhitectura Client-Server\cite{clientserverarchitecture}}
\label{arhitectura-client-server}
\end{figure}

Fiind termeni deja standardizati in aceasta industrie, in continuare vom folosi frontend cand discutam de partea de client si backend cand discutam de partea de server. \\

Intreaga aplicatie este construita folosing limbajul TypeScript, atat pe frontend, cat si pe backend, acesta fiind un superset\footnote{Un limbaj de programare care conține toate caracteristicile unui limbaj dat și a fost extins sau îmbunătățit pentru a include și alte caracteristici.} al limbajulului JavaScript, care ne ofera un sistem sigur pentru tipuri de date.

Tehnologiile pe care le-am folosit pentru frontend sunt:
\begin{itemize}
	\item React - Librarie de JavaScript pentru construirea interfețelor vizuale complexe
	\item Next - Framework\footnote{Un produs software care ofera functionalitati generice pentru construirea mai facila a aplicatiilor} de React, care ne ofera o multime de unelte de optimizare
	\item Material-UI - Librarie de componente UI de React, bazata pe design-ul celor de la Google(Material Design)
	\item Apollo Client - Librarie de React, care ne ajuta la folosirea unui API de tip GraphQL
\end{itemize}

Tehnologiile pe care le-am folosit pentru backend sunt:
\begin{itemize}
	\item Node - Mediu de executie pentru JavaScript, care ne va ajuta la construirea unui backend
	\item Nest - Framework de JavaScript, care ne va ajuta la construirea unui API de tip GraphQL
	\item PostgreSQL - SGBD open-source\footnote{Proiect software ``public'' -- oricine are acces pentru a citi, adauga sau modifica codul sursa, moderat de obicei de un grup restrans}
	\item Prisma - ORM de JavaScript, care ne va ajuta in gestionarea bazei de date
\end{itemize}

In ultimii ani, datorita cresterii utilizarii internetului de pe dispozitivele mobile, dezvoltatorii de aplicatii web au fost nevoiti sa adopte o abordare si un design responsive\footnote{Website-ul poate fi folosit atat de pe ecrane mari(ex. desktop), cat si de pe ecrane mici(ex. telefonul)} cand vine vorba de interfata. Din acest considerent, aplicatia va fi gandita si dezvoltata sa poata fi folosita pe orice dimensiune a ecranului, dar in continuare vom prezenta functionalitatile folosind un ecran mare, deoarece acesta va fi folosit cel mai des de catre utilizatori.

\begin{figure}[H]
\centering
\includegraphics*[width=\columnwidth]{exemplu-design-responsive-desktop}
\caption{Exemplu design responsive - Desktop}
\label{exemplu-design-responsive-desktop}
\end{figure}

\begin{figure}[H]
\centering
\includegraphics*[width=0.4\columnwidth]{exemplu-design-responsive-mobile-menu-closed}
\includegraphics*[width=0.4\columnwidth]{exemplu-design-responsive-mobile-menu-opened}
\caption{Exemplu design responsive - Mobile}
\label{exemplu-design-responsive-mobile}
\end{figure}

\section{Modulul de Autentificare}

Inainte ca utilizatorii sa poata interactiona cu aplicatia, acestia vor trebui sa-si creeze un cont, sau daca au deja un cont sa se autentifice, cu ajutorul modulului de autentificare. Acesta este compus dintr-o pagina cu 2 functionalitati Logare(Login) si Inregistrare(Register).

\begin{figure}[H]
\centering
\includegraphics*[width=0.5\columnwidth]{auth-login}
\caption{Formular Autentificare}
\label{formular-autentificare}
\end{figure}

\begin{figure}[H]
\centering
\includegraphics*[width=0.45\columnwidth]{auth-register1}
\includegraphics*[width=0.45\columnwidth]{auth-register2}
\caption{Formular Inregistrare}
\label{formular-inregistrare}
\end{figure}

Pentru ca un utilizator sa se poate inregistra, acesta va trebui sa completeze urmatoarele campuri:
\begin{itemize}
	\item nume de familie
	\item prenume
	\item initiala tatalui
	\item email - unic printre toti utilizatorii
	\item parola - mai lunga de 6 caractere
	\item avatar(optional)
\end{itemize}

Odata inregistrat, acesta se va putea autentifica folosindu-si email-ul si parola.\\

Aceste date, impreuna cu orice alt formular din aplicatie, vor fi validate de 2 ori, 1 data pe frontend si 1 data pe backend. Validarea pe frontend o vom face din motive de UX, utilizatorul va sti daca datele au un format valid si respecte anumite reguli de validare imediat dupa ce termina un anumit camp de completat. Validarea pe backend o vom face din 2 motive, in primul rand pe frontend nu se pot face validarile care tin cont de date din baza de date, spre exemplu nu putem verifica daca email-ul introdus este unic, iar in al doilea rand, reprezinta o metoda de protectie impotriva celor cu intentii rele, care nu folosesc aplicatia, ci interactioneaza manual cu API-ul acesteia.

Daca oricare dintre cele 2 validari gasesc erori, acestea vor fi afisate utilizatorului exact la campurile corespunzatoare.

\begin{figure}[H]
\centering
\includegraphics*[width=0.45\columnwidth]{exemplu-eroare-validare-frontend}
\includegraphics*[width=0.45\columnwidth]{exemplu-eroare-validare-backend}
\caption{Exemplu Eroare Validare - Frontend(stanga), Backend(dreapta)}
\label{exemplu-eroare-validare}
\end{figure}

\section{Functionalitati administrator}

Odata autentificat, utilizatorul va fi intampinat cu un tablou de bord, de unde isi poate alege una din universitatile la care este inrolat si doreste sa participe, aceasta va fi functionalitatea cea mai des folosita, sau chiar isi poate creea o universitate la care va participa cu rol de administrator, functionalitate destinata conturilor de administrator de universitate.

\begin{figure}[H]
\centering
\includegraphics*[width=\columnwidth]{tablou-de-bord-utilizator}
\caption{Tablou de bord Utilizator}
\label{tablou-de-bord-utilizator}
\end{figure}

Acest tablou este format din 2 componente, o bara de instrumente si lista cu universitati.\\

Bara de instrumente este formata din 2 elemente: 1 buton cu datele utilizatorului, care va redirectiona spre acest tablou de bord, si o iconita de meniu, care va contine un buton pentru delogare.

\begin{figure}[H]
\centering
\includegraphics*[width=\columnwidth]{tablou-de-bord-utilizator-bara-de-instrumente}
\caption{Tablou de bord Utilizator - Bara de instrumente}
\label{tablou-de-bord-utilizator-bara-de-instrumente}
\end{figure}

Lista cu universitati va afisa toate universitatile la care utilizatorul este inrolat, grupate dupa rolul pe care acesta il detine in cadrul universitatii. Iconita de adaugat din dreapta titlului va deschide formular de creare a unei noi universitati avand campurile nume si logo(optional). Fiecare universitate in parte este formata dintr-un buton, care va redirectiona spre tabloul de bord al universitatii, si dintr-un care va deschide un meniu special. In functie de rolul pe care il detinem in cadrul universitatii, acest meniu va contine functionalitati diferite.

\begin{figure}[H]
\centering
\includegraphics*[width=\columnwidth]{tablou-de-bord-utilizator-lista-universitati}
\caption{Tablou de bord Utilizator - Lista Universitati}
\label{tablou-de-bord-utilizator-lista-universitati}
\end{figure}

\begin{figure}[H]
\centering
\includegraphics*[width=0.6\columnwidth]{tablou-de-bord-utilizator-formular-creare-universitate}
\caption{Tablou de bord Utilizator - Formular Creare Universitate}
\label{tablou-de-bord-utilizator-formular-creare-universitate}
\end{figure}

\begin{figure}[H]
\centering
\includegraphics*[width=0.15\columnwidth]{tablou-de-bord-utilizator-meniu-universitate-administrator}
\includegraphics*[width=0.15\columnwidth]{tablou-de-bord-utilizator-meniu-universitate-profesor-student}
\caption{Tablou de bord Utilizator - Meniu Universitate Administrator(stanga), Profesor/Student(dreapta)}
\label{tablou-de-bord-utilizator-meniu-universitate}
\end{figure}

Butonul editare(edit) va deschide un formular de editare al universitatii, cel de parasire(leave) va scoate utilizatorul din lista universitatii, iar cel de sterge(delete) va sterge universitatea si orice date asociate. In cazul butoanelor de parasire si stergere, inainte sa se execute comanda, utilizatorul va fi intrebat daca este sigur ca operatia este cea dorita.

\begin{figure}[H]
\centering
\includegraphics*[width=0.6\columnwidth]{tablou-de-bord-utilizator-formular-editare-universitate}
\caption{Tablou de bord Utilizator - Formular Editare Universitate}
\label{tablou-de-bord-utilizator-formular-editare-universitate}
\end{figure}

\begin{figure}[H]
\centering
\includegraphics*[width=0.45\columnwidth]{tablou-de-bord-utilizator-intrebare-siguranta-stergere}
\includegraphics*[width=0.45\columnwidth]{tablou-de-bord-utilizator-intrebare-siguranta-parasire}
\caption{Tablou de bord Utilizator - Intrebare siguranta stergere(stanga), parasire(dreapta)}
\label{tablou-de-bord-utilizator-intrebare-siguranta}
\end{figure}

\subsection{Structura Universitatii}

Odata selectata o universitate, utilizatorul va fi redirectionat spre tabloul de bord al universitatii. Aici vor fi introduse noi functionalitati in bara de instrumente, in partea de stanga vor aparea un buton cu detaliile universitatii, care va redirectiona spre tabloul de bord al universitatii, si o iconita de meniu care poate fi comutata, pentru a tine meniul de navigare rapida deschis sau a-l inchide. Tot aici, meniul utilizatorului va avea introduse butoane noi in functie de rolul pe care il detine in cadrul universitatii:
\begin{itemize}
	\item Student
		\begin{itemize}
			\item Note(Grades) - vizualizarea rapida a tuturor notelor grupate
		\end{itemize}
	\item Profesor/Administrator
		\begin{itemize}
			\item Set Personal de Intrebari(Question Bank)
		\end{itemize}
	\item Comun
		\begin{itemize}
			\item Activitati Viitoare(Upcoming activities) - Vizualizarea rapida a urmatoarelor activitati in ordine cronologica
		\end{itemize}
\end{itemize}

\begin{figure}[H]
\centering
\includegraphics*[width=\columnwidth]{tablou-de-bord-universitate-bara-de-instrumente}
\caption{Tablou de bord Universitate - Bara de Instrumente}
\label{tablou-de-bord-universitate-bara-de-instrumente}
\end{figure}

\begin{figure}[H]
\centering
\includegraphics*[width=0.4\columnwidth]{tablou-de-bord-universitate-meniu-navigare-rapida}
\caption{Tablou de bord Universitate - Meniu navigare rapida}
\label{tablou-de-bord-universitate-meniu-navigare-rapida}
\end{figure}

\begin{figure}[H]
\centering
\includegraphics*[width=0.4\columnwidth]{tablou-de-bord-universitate-meniu-utilizator-administrator}
\includegraphics*[width=0.4\columnwidth]{tablou-de-bord-universitate-meniu-utilizator-student}
\caption{Tablou de bord Universitate - Meniu Utilizator administrator/profesor(stanga), student(dreapta)}
\label{tablou-de-bord-universitate-meniu-utilizator-administrator}
\end{figure}

Ca si continut, va fi afisata lista de facultati din cadrul universitatii ca si sectiune colapsabila, in care va fi afisata lista de materii din cadrul respectivei facultati. Daca utilizatorul detine rolul de administrator in cadrul universitatii, vor aparea butoane de adaugare, editare sau stergere a facultatilor si materiilor.

\begin{figure}[H]
\centering
\includegraphics*[width=\columnwidth]{tablou-de-bord-universitate-lista-facultati-administrator}
\caption{Tablou de bord Universitate - Lista facultati Administrator}
\label{tablou-de-bord-universitate-lista-facultati-administrator}
\end{figure}

\begin{figure}[H]
\centering
\includegraphics*[width=\columnwidth]{tablou-de-bord-universitate-lista-facultati-student}
\caption{Tablou de bord Universitate - Lista facultati Student}
\label{tablou-de-bord-universitate-lista-facultati-student}
\end{figure}

\begin{figure}[H]
\centering
\includegraphics*[width=0.45\columnwidth]{tablou-de-bord-universitate-formular-creare-facultate}
\includegraphics*[width=0.45\columnwidth]{tablou-de-bord-universitate-formular-editare-facultate}
\caption{Tablou de bord Universitate - Formular creare(stanga), editare(dreapta) Facultate}
\label{tablou-de-bord-universitate-formular-facultate}
\end{figure}

\begin{figure}[H]
\centering
\includegraphics*[width=0.45\columnwidth]{tablou-de-bord-universitate-formular-creare-materie}
\includegraphics*[width=0.45\columnwidth]{tablou-de-bord-universitate-formular-editare-materie}
\caption{Tablou de bord Universitate - Formular creare(stanga), editare(dreapta) Materie}
\label{tablou-de-bord-universitate-formular-materie}
\end{figure}


\begin{figure}[H]
\centering
\includegraphics*[width=0.15\columnwidth]{tablou-de-bord-universitate-meniu-facultate}
\caption{Tablou de bord Universitate - Meniu Facultate/Materie}
\label{tablou-de-bord-universitate-meniu-facultate}
\end{figure}


% \subsection{Managementul utilizatorilor}
% TODO: completeaza dupa ce implementezi in aplicatie

\section{Functionalitati profesor}

Odata selectata o materie, utilizatorul va fi redirectionat spre tabloul de bord al materiei. Aici bara de instrumente isi pastreaza aceleasi functionalitati, iar continutul va fi reprezentat de o lista de sectiuni, fiecare sectiune fiind colapsabila si continand activitati. Profesorul va putea crea atat sectiuni, pentru o mai buna repartizare logica a activitatilor, cat si activitati. Crearea, editarea si stergerea sectiunilor au exact acelasi format ca cel al facultatilor.

\begin{figure}[H]
\centering
\includegraphics*[width=\columnwidth]{tablou-de-bord-materie-lista-sectiuni-profesor}
\caption{Tablou de bord Materie - Lista sectiuni Profesor}
\label{tablou-de-bord-materie-lista-sectiuni-profesor}
\end{figure}

\begin{figure}[H]
\centering
\includegraphics*[width=0.45\columnwidth]{tablou-de-bord-materie-formular-creare-sectiune}
\includegraphics*[width=0.45\columnwidth]{tablou-de-bord-materie-formular-editare-sectiune}
\caption{Tablou de bord Materie - Formular creare(stanga), editare(dreapta) Sectiune}
\label{tablou-de-bord-materie-formular-sectiune}
\end{figure}

\subsection{Crearea activitatilor}

Pentru crearea si editarea activitatilor, formularul se va schimba in functie de tipul de activitate dorit.

Pentru tipurile materiale(resource) si forum, acestea nu vor contine nimic in plus fata de formularul de baza al unei activitati. Orice activitate are urmatoarele campuri de completat:
\begin{itemize}
	\item Nume(Name) - numele activitatii
	\item Descriere(Description)(optional) - descriere sau textul activitatii
	\item Fisiere(Files)(optional) - fisiere ajutatoare pentru parcurgerea activitatii de catre student
\end{itemize}

\begin{figure}[H]
\centering
\includegraphics*[width=0.2\columnwidth]{tablou-de-bord-materie-meniu-alegere-tip-de-activitate}
\caption{Tablou de bord Materie - Meniu alegere tip de Activitate}
\label{tablou-de-bord-materie-meniu-alegere-tip-de-activitate}
\end{figure}

\begin{figure}[H]
\centering
\includegraphics*[width=0.45\columnwidth]{tablou-de-bord-materie-formular-creare-material}
\includegraphics*[width=0.45\columnwidth]{tablou-de-bord-materie-formular-editare-material}
\caption{Tablou de bord Materie - Formular creare(stanga), editare(dreapta) Material/Forum}
\label{tablou-de-bord-materie-formular-material}
\end{figure}

Pentru tipul tema(assignment), se vor mai adauga 2 campuri:
\begin{itemize}
	\item Nota maxima(Max Grade) - nota maxima pe care o poate obtine un student pentru realizarea temei
	\item Termen limita(Deadline) - data si ora pana cand tema poate fi incarcata
\end{itemize}
Nota maxima trebuie sa fie mai mare sau egal cu 0, iar termenul limita trebuie sa reprezinte o data mai mare decat data curenta.

\begin{figure}[H]
\centering
\includegraphics*[width=0.8\columnwidth]{tablou-de-bord-materie-campuri-tema}
\caption{Tablou de bord Materie - Campuri in plus pentru tipul Tema}
\label{tablou-de-bord-materie-campuri-tema}
\end{figure}

\begin{figure}[H]
\centering
\includegraphics*[width=0.32\columnwidth]{tablou-de-bord-materie-camp-alegere-data}
\includegraphics*[width=0.32\columnwidth]{tablou-de-bord-materie-camp-alegere-ora}
\includegraphics*[width=0.32\columnwidth]{tablou-de-bord-materie-camp-alegere-minut}
\caption{Tablou de bord Materie - Camp alegere Timp - data(1), ora(2), minut(3)}
\label{tablou-de-bord-materie-camp-alegere-timp}
\end{figure}

Pentru tipul test(quiz), se vor mai adauga 6 campuri:
\begin{itemize}
	\item Timp deschidere(Time Open) - timp de deschidere al testului
	\item Timp inchidere(Time Close) - timp de inchidere al testului
	\item Timp limita(minute)(Time Limit) - timpul limita maxim admis
	\item Amestecare intrebari(Shuffle Questions) - amesteca aleator ordinea intrebarilor
	\item Amestecare raspunsuri(Shuffle Answers) - amesteca aleator ordinea raspunsurilor fiecarei intrebari
	\item Intrebari(Questions) - setul de intrebari pentru generarea testelor studentilor, iar pentru fiecare intrebare, vom avea 2 campuri
		\begin{itemize}
			\item Ordinea - ordinea in lista de intrebari
			\item Nota maxima - nota maxima pe care o poate obtine un student pentru rapunsul corect la intrebare
		\end{itemize}
\end{itemize}
Timpul de deschidere si cel de inchidere trebuie sa reprezinte o data mai mare decat data curenta, timpul de inchidere trebuie sa fie dupa timpul de deschidere, timpul limita trebuie sa fie mai mare decat 0, iar pentru fiecare intrebare in parte, nota maxima trebuie sa fie mai mare sau egala cu 0.

\begin{figure}[H]
\centering
\includegraphics*[width=0.7\columnwidth]{tablou-de-bord-materie-campuri-test}
\caption{Tablou de bord Materie - Campuri in plus pentru tipul Test}
\label{tablou-de-bord-materie-campuri-test}
\end{figure}

\begin{figure}[H]
\centering
\includegraphics*[width=\columnwidth]{tablou-de-bord-materie-formular-adaugare-intrebari}
\caption{Tablou de bord Materie - Formular adaugare intrebari}
\label{tablou-de-bord-materie-formular-adaugare-intrebari}
\end{figure}

\subsection{Accesarea activitatilor}

Odata cu selectarea unei activitati, utilizatorul va fi redirectionat catre un tablou de bord dedicat fiecarui tip de activitate. Orice activitate va avea afisat la inceput detaliile de baza: nume, descriere, fisiere, dupa care va continua cu detalii diferite in functie de tip.

\begin{figure}[H]
\centering
\includegraphics*[width=\columnwidth]{tablou-de-bord-activitate-detalii-de-baza}
\caption{Tablou de bord Activitate - Detalii de baza}
\label{tablou-de-bord-activitate-detalii-de-baza}
\end{figure}

In cazul temelor, se vor afisa cele 2 campuri in plus intr-un tabel, Detalii Tema, iar sub tabel va fi lista cu studentii participanti, care au incarcat rezolvari la teme, impreuna cu detaliile lor: avatar-ul, punctajul, numele si ultima data de incarcare.

\begin{figure}[H]
\centering
\includegraphics*[width=\columnwidth]{tablou-de-bord-activitate-detalii-specifice-temelor}
\caption{Tablou de bord Activitate - Detalii specifice temelor}
\label{tablou-de-bord-activitate-detalii-specifice-temelor}
\end{figure}

Odata apasat unul dintre studenti, utilizatorul va fi redirectionat catre un tablou de bord dedicat notarii studentului, unde vor fi afisate datele despre tema, rezolvarea incarcata de acesta si un formular de notare in dreapta paginii.

\begin{figure}[H]
\centering
\includegraphics*[width=\columnwidth]{tablou-de-bord-notare-tema-student}
\caption{Tablou de bord - Notare tema student}
\label{tablou-de-bord-notare-tema-student}
\end{figure}

In cazul testelor, se vor afisa intr-un tabel detaliile specifice testului, iar sub acesta vor exista 2 sectiuni, prima este cea care contine detalii despre intrebarile care apartin de test, aceasta fiind o sectiune colapsabila, iar cea de-a doua este lista cu studentii care au participat la test.

\begin{figure}[H]
\centering
\includegraphics*[width=\columnwidth]{tablou-de-bord-activitate-detalii-specifice-testelor}
\caption{Tablou de bord Activitate - Detalii specifice testelor}
\label{tablou-de-bord-activitate-detalii-specifice-testelor}
\end{figure}

\begin{figure}[H]
\centering
\includegraphics*[width=\columnwidth]{tablou-de-bord-activitate-intrebari-test}
\caption{Tablou de bord Activitate - Intrebari test}
\label{tablou-de-bord-activitate-intrebari-test}
\end{figure}

\begin{figure}[H]
\centering
\includegraphics*[width=\columnwidth]{tablou-de-bord-activitate-lista-studenti-test}
\caption{Tablou de bord Activitate - Lista studenti Test}
\label{tablou-de-bord-activitate-lista-studenti-test}
\end{figure}

Odata apasat unul dintre studenti, utilizatorul va fi redirectionat catre un tablou de bord dedicat notarii studentului, dar deoarece notarea testelor se face automat, aceasta pagina este strict pentru vizionarea testului. Ca si continut se vor afisa intrebarile si raspunsurile la acestea, respectiv nota obtinuta la fiecare intrebare, iar in dreapta se va regasi o metoda de navigare rapida prin intrebarile testului.

\begin{figure}[H]
\centering
\includegraphics*[width=\columnwidth]{tablou-de-bord-notare-test-student}
\caption{Tablou de bord - Notare test student}
\label{tablou-de-bord-notare-test-student}
\end{figure}

In cazul forumurilor, se vor afisa comentariile adaugate de participanti, orice rol este permis.

\begin{figure}[H]
\centering
\includegraphics*[width=\columnwidth]{tablou-de-bord-forum}
\caption{Tablou de bord Activitate - Forum}
\label{tablou-de-bord-forum}
\end{figure}

\begin{figure}[H]
\centering
\includegraphics*[width=0.5\columnwidth]{tablou-de-bord-activitate-adaugare-comentariu-forum}
\caption{Tablou de bord Activitate - Adaugare comentariu Forum}
\label{tablou-de-bord-activitate-adaugare-comentariu-forum}
\end{figure}

\subsection{Crearea setului personal de intrebari}

La selectarea Setului personal de intrebari din meniul, utilizatorul va fi redirectionat catre tabloul de bordcu intrebari. Acesta va contine o lista de categorii de intrebari, pentru o mai buna repartizare logica a acestora, iar fiecare categorie va fi o sectiune colapsabila care va contine intrebarile specifice.

\begin{figure}[H]
\centering
\includegraphics*[width=\columnwidth]{tablou-de-bord-set-intrebari}
\caption{Tablou de bord Set Intrebari}
\label{tablou-de-bord-set-intrebari}
\end{figure}

\begin{figure}[H]
\centering
\includegraphics*[width=0.45\columnwidth]{tablou-de-bord-set-intrebari-creare-categorie}
\includegraphics*[width=0.45\columnwidth]{tablou-de-bord-set-intrebari-editare-categorie}
\caption{Tablou de bord Set Intrebari - Formular creare(stanga), editare(dreapta) Categorie}
\label{tablou-de-bord-set-intrebari-categorie}
\end{figure}

Formularul de creare a unei intrebari cuprinde urmatoarele campuri:
\begin{itemize}
	\item Tip de activitate(Type) - care poate fi cu raspuns unic(Single Choice), sau cu raspuns multiplu(Multiple Choice)
	\item Nume(Name) - numele intrebarii
	\item Text - textul intrebarii
	\item Raspunsuri(Answers) - raspunsurile intrebarii, iar pentru fiecare raspunsu trebuie completate 2 campuri
		\begin{itemize}
			\item Ordine - ordinea din lista de raspunsuri
			\item Text - textul raspunsului
			\item Procentaj/Fractiune - ponderea pe care o are raspunsul respectiv pentru intrebare, poate lua valori intre -100 si 100
		\end{itemize}
\end{itemize}

\begin{figure}[H]
\centering
\includegraphics*[width=0.7\columnwidth]{tablou-de-bord-set-intrebari-formular-intrebare}
\caption{Tablou de bord Set Intrebari - Formular Intrebare}
\label{tablou-de-bord-set-intrebari-formular-intrebare}
\end{figure}

Pentru o calculare automata corecta a notelor de la teste, exista cateva indicatii care trebuie respectate.
\begin{itemize}
	\item Procentajul poate lua valori intre -100 si 100
	\item Pentru intrebarile cu raspuns unic, doar 1 singur raspuns ar trebui sa aiba procentajul = 100, restul ar trebui sa fie = 0
	\item Pentru intrebarile cu raspuns multiplu, suma tuturor procentajelor positive ar trebui sa fie = 100
	\item Pentru intrebarile cu raspuns multiplu, suma tuturor procentajelor negative ar trebui sa fie = -100
\end{itemize}

\begin{figure}[H]
\centering
\includegraphics*[width=0.7\columnwidth]{tablou-de-bord-set-intrebari-indicatii-notare-automata}
\caption{Tablou de bord Set Intrebari - Indicatii Notare Automata}
\label{tablou-de-bord-set-intrebari-indicatii-notare-automata}
\end{figure}

\section{Functionalitati student}

Ajungand in tabloul de bord al materiei, utilizatorii cu rol de student vor vedea aceleasi informatii ca si cei cu rol de profesor, in afara de butoanele de creare, editare sau stergere ale sectiunilor si activitatilor.

\begin{figure}[H]
\centering
\includegraphics*[width=0.7\columnwidth]{tablou-de-bord-materie-lista-sectiuni-student}
\caption{Tablou de bord Materie - Lista Sectiuni Student}
\label{tablou-de-bord-materie-lista-sectiuni-student}
\end{figure}

\subsection{Participarea la activitati}

Odata selectata o activitate, utilizatorul va fi redirectionat catre tabloul de bord al activitatii, care va afisa ca si in cazul utilizatorilor cu rol de profesor, detalii de baza despre activitate, dar si detalii specifice studentului. Diferentele notabile sunt pentru tipurile tema si test.

In cazul temelor, studentului i se vor afisa detalii personale legate de tema respectiva, nota, ultima data de incarcare, si daca timpul limita nu s-a scurs, o modalitate de incarcare a temei.

\begin{figure}[H]
\centering
\includegraphics*[width=\columnwidth]{tablou-de-bord-activitate-incarcare-tema}
\caption{Tablou de bord Activitate - Incarcare tema}
\label{tablou-de-bord-activitate-incarcare-tema}
\end{figure}

In cazul testelor, ca si in cazul temelor, studentului i se vor afisa detalii personale legate de testul respectiv, nota, timp deschidere test, timp trimitere test pentru corectare.

\begin{figure}[H]
\centering
\includegraphics*[width=\columnwidth]{tablou-de-bord-activitate-detalii-test-student}
\caption{Tablou de bord Activitate - Detalii test Student}
\label{tablou-de-bord-activitate-detalii-test-student}
\end{figure}

In functie de timpul curent si detaliile testului, utilizatorului i se va afisa una dintre cele 5 stari ale butonului de interactionare:
\begin{itemize}
	\item Nimic - testul nu este activ inca sau testul a expirat, utilizatorul nu si-a inceput incercarea
	\item Incepe - testul este activ, utilizatorul nu si-a inceput incercarea
	\item Continua - testul este activ, utilizatorul si-a inceput incercarea
	\item Verifica(blocat) - testul este activ, utilizatorul a trimis rezolvarea pentru notare
	\item Verifica - testul a expirat, utilizatorul si-a inceput incercarea
\end{itemize}

Butoanele Incepe si Continua vor redirectiona studentul spre incercarea curenta a testului. Starea testului, raspunsurile la intrebari, sunt salvate incremental, in timp ce studentul rezolva testul, in cazul unei urgente sau deconectari, studentul nu va pierde tot progresul pe care l-a facut.

Tabloul de bord al testului activ este format din meniul pentru navigare rapida in dreapta paginii, iar ca si continut se vor afisa intrebarile, una cate una, detalii despre aceasta, si timpul ramas din incercare. Toate celelalte butoane si meniuri din bara de instrumente vor fi ascunse, pentru cat mai putine distrageri si/sau apasari din greseala.

\begin{figure}[H]
\centering
\includegraphics*[width=\columnwidth]{tablou-de-bord-test-activ}
\caption{Tablou de bord Test activ}
\label{tablou-de-bord-test-activ}
\end{figure}

\begin{figure}[H]
\centering
\includegraphics*[width=0.7\columnwidth]{tablou-de-bord-test-activ-intrebare-raspuns-unic}
\caption{Tablou de bord Test activ - Intrebare raspuns unic}
\label{tablou-de-bord-test-activ-intrebare-raspuns-unic}
\end{figure}

\begin{figure}[H]
\centering
\includegraphics*[width=0.7\columnwidth]{tablou-de-bord-test-activ-intrebare-raspuns-multiplu}
\caption{Tablou de bord Test activ - Intrebare raspuns multiplu}
\label{tablou-de-bord-test-activ-intrebare-raspuns-multiplu}
\end{figure}

Dupa ce testul a expirat, utilizatorul isi poate verifica raspunsurile la intrebari si notele.

\begin{figure}[H]
\centering
\includegraphics*[width=\columnwidth]{tablou-de-bord-verificare-test}
\caption{Tablou de bord Verificare Test}
\label{tablou-de-bord-verificare-test}
\end{figure}

% \subsection{Vizualizarea notelor}
% TODO: do this when you have time
% \subsection{Vizualizarea urmatoarelor activitati}
% TODO: do this when you have time

% \section{Baza de date}
% TODO: do this when you have time - logical separation

% \section{Securitate}
% TODO: do this when you have time - jwt, roles, scopes

\chapter*{Concluzii}

Proiectul implementeaza functionalitatile cele mai importante pentru desfasurarea activitatii de predare/testare in mediu online. Aceasta metoda nu are rolul de a inlocui desfasurarea activitii in mediu fizic, ci de a o imbunatati. Consider ca integrarea unei platforme de e-learning in cadrul oricarei universitati vine cu beneficii atat pentru studenti, cat si pentru profesori. Trecerea in mediu online a unor tipuri de activitati poate salva timp, efort, chiar si bani pentru laturile implicate.

\addcontentsline{toc}{chapter}{Concluzii}

\bibliographystyle{unsrt}
\bibliography{referinte}

% \appendix

\end{document}

