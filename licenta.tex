\documentclass[12pt, a4paper, oneside, romanian]{teza-upb}
\setcounter{secnumdepth}{3}
\setcounter{tocdepth}{3}
\usepackage{babel}
\usepackage{color}
\usepackage{graphicx}
\usepackage{float}
\graphicspath{ {./imagini/} }
\usepackage{indentfirst}
\usepackage[
  bookmarks,
  bookmarksopen=true,
  pdftitle={licenta},
  linktocpage,
  unicode]{hyperref}


\singlespacing

%%%%%%%%%%%%%%%%%%%%%%%%%%%%%%%%%%%%%%%%%%%%%%%%%%%%%%%%cod colorat
\usepackage{listings}
\usepackage{color}
 
\definecolor{codegreen}{rgb}{0,0.6,0}
\definecolor{codegray}{rgb}{0.5,0.5,0.5}
\definecolor{codepurple}{rgb}{0.58,0,0.82}
\definecolor{backcolour}{rgb}{0.95,0.95,0.92}

\lstdefinelanguage{JavaScript}{
  keywords={typeof, new, true, false, catch, function, return, null, catch, switch, var, if, in, while, do, else, case, break},
  ndkeywords={class, export, boolean, throw, implements, import, this},
  sensitive=false,
  comment=[l]{//},
  morecomment=[s]{/*}{*/},
  morestring=[b]',
  morestring=[b]"
}

\lstset{language=JavaScript}

\lstdefinestyle{mystyle}{
	language=JavaScript,
    commentstyle=\color{codegreen},
    keywordstyle=\color{blue},
  	ndkeywordstyle=\color{blue},
    numberstyle=\tiny\color{codegray},
  	identifierstyle=\color{black},
    stringstyle=\color{codepurple},
	basicstyle=\ttfamily\scriptsize,
    breakatwhitespace=false,         
    breaklines=true,                 
    captionpos=b,                    
    keepspaces=true,                 
    numbers=left,                    
    numbersep=5pt,                  
    showspaces=false,                
    showstringspaces=false,
    showtabs=false,                  
    tabsize=2
}
 
\lstset{style=mystyle}
%%%%%%%%%%%%%%%%%%%%%%%%%%%%%%%%%%%%%%%%%%%%%%%

\begin{document}

\author{Albert-Lucian LUŢĂ}

\title{Includerea instrumentelor de predare-testare într-o aplicație e-learning}

\facultatea{Facultatea de Electronică, Telecomunicații și Tehnologia Informației}
\tiplucrare{diploma}
\domeniu{Calculatoare și Tehnologia Informației}
\catedra{Telecomunicații}
\campus{Leu} 
\program{Ingineria Informației}
\titlulobtinut{Inginer}
\director{Ş.L. Dr. Ing. Elena Cristina STOICA} 

\submissionmonth{Septembrie} 
\submissionyear{2021} 

\beforepreface
\newpage
\listoffigures
% \listoftables
\abbreviations{
	API = Aplication Programming Interface(Interfață de programare a aplicației)

	ORM = Object-relational Mapping(Cartografiere obiect-relațională)

	SGBD = Sistem de gestiune a bazei de date

	TCP = Transmission Control Protocol(Protocol de control al transmisiei)

	UI = User Interface(Interfață cu utilizatorul)

	UX = User Experience(Experiența utilizatorului)
}
\afterpreface 

\chapter*{Introducere}
\addcontentsline{toc}{chapter}{Introducere}

\section{Scopul Proiectului}

Proiectul are ca scop dezvoltarea unei aplicații web e-learning, cu ajutorul căreia universitățile să-și poată desfășura activitatea de predare/testare și în mediu online. Această metodă nu are rolul de a înlocui desfășurarea activităţii în mediu fizic, ci de a o îmbunătăţii. Spre exemplu, studenții pot avea acces de oriunde și oricând la materialele și notele fiecărei materii, iar profesorii pot primi temele și proiectele pentru curs, seminar sau laborator online, fără a  mai fi nevoie să transporte foile de hârtie ale fiecărui student. Cu toate acestea, mediul online prezintă dezavantaje în momentul testării/examinării cunoștințelor proprii ale studenților, deoarece comunicarea online este foarte greu sau chiar imposibil de oprit. Din acest motiv, consider ca examinările fizice sunt cea mai bună modalitate în acest caz, iar cele online pot fi păstrate pentru simulări, exerciții propuse sau variante din ani trecuți ai examenului.

\section{Descrierea proiectului}

Aplicația permite crearea și managementul unei universități. În cadrul universității pot exista 3 tipuri de utilizatori: utilizator cu rol de administrator, utilizator cu rol de profesor și utilizator cu rol de student, fiecare având acces la funcționalități diferite.
\begin{itemize}
	\item Student
		\begin{itemize}
			\item Accesarea materialelor(cursuri, seminarii, laboratoare)
			\item Accesarea temelor și încărcarea rezolvărilor
			\item Participarea la teste și simulări
			\item Participarea la discuții pe forum
			\item Vizualizarea notelor
			\item Vizualizarea viitoarelor activități
		\end{itemize}
	\item Profesor
		\begin{itemize}
			\item Încărcarea materialelor
			\item Crearea temelor
			\item Crearea testelor/simulărilor
			\item Notarea studenților
			\item Participarea la discuții pe forum
			\item Împărțirea activităților(materiale, teme, teste, forum) pe secțiuni
			\item Crearea setului personal de întrebări și răspunsuri pentru teste
		\end{itemize}
	\item Administrator
		\begin{itemize}
			\item Managementul facultăților
			\item Managementul materiilor
			\item Managementul utilizatorilor
		\end{itemize}
\end{itemize}

\chapter{Proiectare}

Înainte de a trece la implementarea propriu-zisă a proiectului, trebuie să înțelegem foarte bine cerințele și metodele prin care acesta va fi construit. Această treapta preliminară implementării se numește proiectare, și a devenit din ce în ce mai importantă cu cât programele software au devenit mai complexe.

Proiectarea software reprezintă procesul de găsire a soluțiilor conceptuale pentru realizarea unei aplicații, pe baza unui set de cerințe și așteptări. Acest pas poate reduce costurile și timpul de implementare, în cazul unor neînțelegeri inițiale sau a unor schimbări de cerințe, și ușurează procesul de estimare al timpului și costului de realizare al întregului proiect.

\begin{figure}[H]
\centering
\includegraphics*[width=0.55\columnwidth]{proiectare-software-model-cascada}
\caption{Proiectare Software - Model Cascadă, Sursă: \cite{proiectaresoftwaremodelcascada}}
\label{proiectare-software-model-cascada}
\end{figure}

\section{Funcționalități administrator}

Functionalităţile utilizatorilor cu rol de administrator sunt următoarele:
\begin{itemize}
	\item Managementul facultăților
	\item Managementul materiilor
	\item Managementul utilizatorilor
\end{itemize}

\subsection{Structura universității}

Înainte de a putea face managementul utilizatorilor, administratorul este nevoit să definească structura universității - facultățile și materiile componente. Acesta va putea adăuga facultăți, iar pentru fiecare dintre acestea va putea adăuga materiile asociate.

\begin{figure}[H]
\centering
\includegraphics*[width=0.5\columnwidth]{diagrama-use-case-creeaza-structura-universitatii}
\caption{Diagramă Caz de utilizare - Creează Structura Universității}
\label{diagrama-use-case-creeaza-structura-universitatii}
\end{figure}


\begin{figure}[H]
\centering
\includegraphics*[width=0.5\columnwidth]{diagrama-activitate-creeaza-structura-universitatii}
\caption{Diagramă Activitate - Creează Structura Universității}
\label{diagrama-activitate-creeaza-structura-universitatii}
\end{figure}

\subsection{Managementul utilizatorilor}

Odată stabilită structura universității, administratorul va putea adăuga utilizatori, le va putea atribui un anumit rol și îi va putea repartiza la facultățile și materiile la care vor avea acces.

\begin{figure}[H]
\centering
\includegraphics*[width=0.525\columnwidth]{diagrama-use-case-adauga-utilizator}
\caption{Diagramă Caz de utilizare - Adaugă Utilizator}
\label{diagrama-use-case-adauga-utilizator}
\end{figure}


\begin{figure}[H]
\centering
\includegraphics*[width=0.6\columnwidth]{diagrama-activitate-adauga-utilizator}
\caption{Diagramă Activitate - Adaugă Utilizator}
\label{diagrama-activitate-adauga-utilizator}
\end{figure}

\section{Funcționalități profesor}

Funcţionalităţile utilizatorilor cu rol de profesor sunt următoarele:
\begin{itemize}
	\item Încărcarea materialelor
	\item Crearea temelor
	\item Crearea testelor/simulărilor
	\item Notarea studenților
	\item Împărțirea activităților(materiale, teme, teste, forum) pe secțiuni
	\item Crearea setului personal de întrebări și răspunsuri pentru teste
\end{itemize}

\subsection{Crearea activităților}

Activitățile pot fi de 4 tipuri: materiale, temă, test și forum. Pentru crearea unei activități vom avea nevoie atât de detaliile de bază, pe care le vom găsi la orice tip de activitate, cât și de detaliile specifice fiecărui tip în parte. Pentru o mai bună repartizare logică a activităților, acestea vor fi împărțite pe secțiuni.

\begin{figure}[H]
\centering
\includegraphics*[width=0.9\columnwidth]{diagrama-use-case-creeaza-activitate}
\caption{Diagramă Caz de utilizare - Creează Activitate}
\label{diagrama-use-case-creeaza-activitate}
\end{figure}


\begin{figure}[H]
\centering
\includegraphics*[width=0.65\columnwidth]{diagrama-activitate-creeaza-activitate}
\caption{Diagramă Activitate - Creează Activitate}
\label{diagrama-activitate-creeaza-activitate}
\end{figure}

\subsection{Notarea studenţilor}

Pentru notare, profesorul este nevoit în primul rând să selecteze activitatea pentru care se va desfășura procesul. Odată aleasă activitatea, acesta va putea alege unul dintre studenții care au participat la aceasta și îi va putea nota.

\begin{figure}[H]
\centering
\includegraphics*[width=0.65\columnwidth]{diagrama-use-case-noteaza-student}
\caption{Diagramă Caz de utilizare - Notează Student}
\label{diagrama-use-case-noteaza-student}
\end{figure}


\begin{figure}[H]
\centering
\includegraphics*[width=0.45\columnwidth]{diagrama-activitate-noteaza-student}
\caption{Diagramă Activitate - Notează Student}
\label{diagrama-activitate-noteaza-student}
\end{figure}

\subsection{Crearea setului personal de întrebări}

Pentru a face mai facilă refolosirea întrebărilor de la un an la altul, profesorii își vor putea crea un set personal de întrebări pe care îl va folosi la generarea testelor. Acesta va putea crea întrebări de diferite tipuri, repartizate în funcție de o categorie, adăuga răspunsuri și marca pe cele corecte și greșite.

\begin{figure}[H]
\centering
\includegraphics*[width=0.3\columnwidth]{diagrama-use-case-creeaza-set-personal-de-intrebari}
\caption{Diagramă Caz de utilizare - Creează Set Personal de Întrebări}
\label{diagrama-use-case-creeaza-set-personal-de-intrebari}
\end{figure}


\begin{figure}[H]
\centering
\includegraphics*[width=0.75\columnwidth]{diagrama-activitate-creeaza-set-personal-de-intrebari}
\caption{Diagramă Activitate - Creează Set Personal de Întrebări}
\label{diagrama-activitate-creeaza-set-personal-de-intrebari}
\end{figure}

\section{Funcționalități student}

Funcţionalităţile utilizatorilor cu rol de student sunt următoarele:
\begin{itemize}
	\item Accesarea materialelor(cursuri, seminarii, laboratoare)
	\item Accesarea temelor și încărcarea rezolvărilor
	\item Participarea la teste și simulări
	\item Vizualizarea notelor
\end{itemize}

\subsection{Accesarea detaliilor fiecărei activități}

După ce profesorul a creat structura materiei, secțiuni și activități, studenții vor avea acces la acestea. Ei își vor putea alege activitatea la care vor să ia parte, și vizualiza detaliile despre aceasta.

\begin{figure}[H]
\centering
\includegraphics*[width=\columnwidth]{diagrama-use-case-acceseaza-activitate}
\caption{Diagramă Caz de utilizare - Accesează Activitate}
\label{diagrama-use-case-acceseaza-activitate}
\end{figure}


\begin{figure}[H]
\centering
\includegraphics*[width=0.6\columnwidth]{diagrama-activitate-acceseaza-activitate}
\caption{Diagramă Activitate - Accesează Activitate}
\label{diagrama-activitate-acceseaza-activitate}
\end{figure}

\subsection{Încărcarea rezolvărilor pentru teme}

Pentru activitățile de tip temă, aceștia își vor putea încărca rezolvările și vizualiza nota, după ce tema a fost corectată și notată de către profesor.

\begin{figure}[H]
\centering
\includegraphics*[width=0.5\columnwidth]{diagrama-use-case-incarca-rezolvare-tema}
\caption{Diagramă Caz de utilizare - Încarcă Rezolvare Temă}
\label{diagrama-use-case-incarca-rezolvare-tema}
\end{figure}


\begin{figure}[H]
\centering
\includegraphics*[width=0.55\columnwidth]{diagrama-activitate-incarca-rezolvare-tema}
\caption{Diagramă Activitate - Încarcă Rezolvare Temă}
\label{diagrama-activitate-incarca-rezolvare-tema}
\end{figure}


\subsection{Participarea la teste}

Pentru activitățile de tip test, studenții vor putea interacționa în diferite moduri, în funcție de data curentă și detaliile testului. Vor avea posibilitatea de a începe o încercare sau a o continua, dacă testul este activ și verifica rezultatele întrebărilor, dacă testul a expirat. 

\begin{figure}[H]
\centering
\includegraphics*[width=0.75\columnwidth]{diagrama-use-case-participa-la-test}
\caption{Diagramă Caz de utilizare - Participă la Test}
\label{diagrama-use-case-participa-la-test}
\end{figure}


\begin{figure}[H]
\centering
\includegraphics*[width=0.75\columnwidth]{diagrama-activitate-participa-la-test}
\caption{Diagramă Activitate - Participă la Test}
\label{diagrama-activitate-participa-la-test}
\end{figure}

\section{Funcționalități comune}

Funcţionalităţile comune sunt următoarele:
\begin{itemize}
	\item Participarea la discuţii pe forum
\end{itemize}

\subsection{Participarea la discuţii pe forum}

Indiferent de rolul pe care un utilizator îl are în universitate, acesta va putea participa la discuţii pe forum. Cu toate acestea, doar administratorii sau profesorii pot crea discuţii noi.

\begin{figure}[H]
\centering
\includegraphics*[width=0.45\columnwidth]{diagrama-use-case-participa-la-forum}
\caption{Diagramă Caz de utilizare - Participă la Forum}
\label{diagrama-use-case-participa-la-forum}
\end{figure}

\begin{figure}[H]
\centering
\includegraphics*[width=0.5\columnwidth]{diagrama-activitate-participa-la-forum}
\caption{Diagramă Activitate - Participă la Forum}
\label{diagrama-activitate-participa-la-forum}
\end{figure}

\chapter{Implementare}

Fiind o aplicaţie web, am folosit arhitectura standard de construire a acestora "client-server". Aceasta împarte aplicaţia în două părţi, fiecare având un rol separat, comunicând între ele prin conexiuni TCP. Partea de client se referă la interfaţa vizuală cu care interacţionează un utilizator în momentul folosirii, iar partea de server se referă la codul care rulează pe un calculator, de obicei în cloud\footnote{Ansamblu distribuit de servicii de calcul}, care răspunde cererilor primite de la client, cel din urma stocând datele necesare rulării aplicaţiei, de cele mai multe ori într-o bază de date.

\begin{figure}[H]
\centering
\includegraphics*[width=0.7\columnwidth]{arhitectura-client-server}
\caption{Arhitectură Client-Server}
\label{arhitectura-client-server}
\end{figure}

Întreaga aplicaţie este construită folosind limbajul TypeScript, atât pe partea de client, cât şi pe partea de server, acesta fiind un superset\footnote{Un limbaj de programare care conține toate caracteristicile unui limbaj dat și a fost extins sau îmbunătățit pentru a include și alte caracteristici} al limbajului JavaScript, care ne oferă un sistem sigur pentru tipuri de date.

Tehnologiile pe care le-am folosit pe partea de client sunt:
\begin{itemize}
	\item React - Librărie de JavaScript pentru construirea interfețelor vizuale complexe
	\item Next - Framework\footnote{Un produs software care oferă funcţionalităţi generice pentru construirea mai facilă a aplicaţiilor} de React, care ne oferă o mulţime de unelte de optimizare
	\item Material-UI - Librărie de componente UI de React, bazată pe design-ul celor de la Google(Material Design)
	\item Apollo Client - Librărie de React, care ne ajută la folosirea unui API de tip GraphQL
\end{itemize}

Tehnologiile pe care le-am folosit pe partea de server sunt:
\begin{itemize}
	\item Node - Mediu de execuţie pentru JavaScript, care ne va ajuta la construirea unui server
	\item Nest - Framework de JavaScript, care ne va ajuta la construirea unui API de tip GraphQL
	\item PostgreSQL - SGBD open-source\footnote{Proiect software ``public'' -- oricine are acces pentru a citi, adăuga sau modifica codul sursă, moderat de obicei de un grup restrâns}
	\item Prisma - ORM de JavaScript, care ne va ajuta în gestionarea bazei de date
\end{itemize}

În ultimii ani, datorită creşterii utilizării internetului pe dispozitivele mobile, dezvoltatorii de aplicaţii web au fost nevoiţi să adopte o abordare şi un design flexibil\footnote{Website-ul poate fi folosit atât de pe ecrane mari(ex. desktop), cât şi de pe ecrane mici(ex. telefonul mobil)} când vine vorba de interfaţă. Din acest considerent, aplicaţia va fi gândită şi dezvoltata să poată fi folosită pe orice dimensiune a ecranului, dar în continuare vom prezenta funcţionalităţile folosind un ecran mare, deoarece acesta va fi folosit cel mai des de către utilizatori.

\begin{figure}[H]
\centering
\includegraphics*[width=\columnwidth]{exemplu-design-responsive-desktop}
\caption{Exemplu design flexibil - Desktop}
\label{exemplu-design-responsive-desktop}
\end{figure}

\begin{figure}[H]
\centering
\includegraphics*[width=0.4\columnwidth]{exemplu-design-responsive-mobile-menu-closed}
\includegraphics*[width=0.4\columnwidth]{exemplu-design-responsive-mobile-menu-opened}
\caption{Exemplu design flexibil - Telefon Mobil}
\label{exemplu-design-responsive-mobile}
\end{figure}

\section{Modulul de Autentificare}

Înainte ca utilizatorii să poată interacţiona cu aplicaţia, aceştia vor trebui să-şi creeze un cont, sau dacă au deja un cont să se autentifice, cu ajutorul modulului de autentificare. Acesta este compus dintr-o pagină cu 2 funcţionalităţi Autentificare şi Înregistrare.

\begin{figure}[H]
\centering
\includegraphics*[width=0.5\columnwidth]{auth-login}
\caption{Formular Autentificare}
\label{formular-autentificare}
\end{figure}

\begin{figure}[H]
\centering
\includegraphics*[width=0.45\columnwidth]{auth-register1}
\includegraphics*[width=0.45\columnwidth]{auth-register2}
\caption{Formular Înregistrare}
\label{formular-inregistrare}
\end{figure}

Pentru ca un utilizator să se poate înregistra, acesta va trebui să completeze următoarele câmpuri:
\begin{itemize}
	\item nume de familie
	\item prenume
	\item iniţiala tatălui
	\item email - unic printre toţi utilizatorii
	\item parola - mai lunga de 6 caractere
	\item avatar(opţional)
\end{itemize}

Odată înregistrat, acesta se va putea autentifica folosindu-şi email-ul şi parola.\\

Aceste date, împreuna cu orice alt formular din aplicaţie, vor fi validate de 2 ori, 1 dată pe partea de client şi 1 dată pe partea de server. Validarea pe partea de client o vom face din motive de UX, utilizatorul va şti dacă datele au un format valid şi respecta anumite reguli de validare imediat după ce termină un anumit câmp de completat. Validarea pe partea de server o vom face din 2 motive, în primul rând pe partea de client nu se pot face validările care ţin cont de date din baza de date, spre exemplu nu putem verifica dacă email-ul introdus este unic, iar în al doilea rând, reprezintă o metodă de protecţie împotriva celor cu intenţii rele, care nu folosesc aplicaţia, ci interacţionează manual cu API-ul acesteia.

Dacă oricare dintre cele 2 validări găsesc erori, acestea vor fi afişate utilizatorului exact la câmpurile corespunzătoare.

\begin{figure}[H]
\centering
\includegraphics*[width=0.45\columnwidth]{exemplu-eroare-validare-frontend}
\includegraphics*[width=0.45\columnwidth]{exemplu-eroare-validare-backend}
\caption{Exemplu Eroare Validare - Client(stânga), Server(dreapta)}
\label{exemplu-eroare-validare}
\end{figure}

\section{Funcţionalităţi administrator}

Odată autentificat, utilizatorul va fi întâmpinat cu un tablou de bord, de unde îşi poate alege una din universităţile la care este înrolat şi doreşte să participe, aceasta va fi funcţionalitatea cea mai des folosită, sau chiar îşi poate creea o universitate la care va participa cu rol de administrator, funcţionalitate destinată conturilor de administrator de universitate.

\begin{figure}[H]
\centering
\includegraphics*[width=\columnwidth]{tablou-de-bord-utilizator}
\caption{Tablou de bord Utilizator}
\label{tablou-de-bord-utilizator}
\end{figure}

Acest tablou este format din 2 componente, o bară de instrumente şi lista cu universităţi.\\

Bara de instrumente este formată din 2 elemente: 1 buton cu datele utilizatorului, care va redirecţiona spre acest tablou de bord, şi o iconiţă de meniu, care va conţine un buton pentru deconectare.

\begin{figure}[H]
\centering
\includegraphics*[width=\columnwidth]{tablou-de-bord-utilizator-bara-de-instrumente}
\caption{Tablou de bord Utilizator - Bară de instrumente}
\label{tablou-de-bord-utilizator-bara-de-instrumente}
\end{figure}

Lista cu universităţi va afişa toate universităţile la care utilizatorul este înrolat, grupate după rolul pe care acesta îl deţine în cadrul universităţii. Iconiţa de adăugat din dreapta titlului va deschide formularul de creare a unei noi universităţi, având câmpurile nume şi siglă(opţional). Fiecare universitate este formată dintr-un buton, care va redirecţiona spre tabloul de bord al universităţii, şi dintr-un buton care va deschide un meniu special. În funcţie de rolul pe care utilizatorul îl deţine în cadrul universităţii, acest meniu va conţine funcţionalităţi diferite.

\begin{figure}[H]
\centering
\includegraphics*[width=\columnwidth]{tablou-de-bord-utilizator-lista-universitati}
\caption{Tablou de bord Utilizator - Listă Universităţi}
\label{tablou-de-bord-utilizator-lista-universitati}
\end{figure}

\begin{figure}[H]
\centering
\includegraphics*[width=0.6\columnwidth]{tablou-de-bord-utilizator-formular-creare-universitate}
\caption{Tablou de bord Utilizator - Formular Creare Universitate}
\label{tablou-de-bord-utilizator-formular-creare-universitate}
\end{figure}

\begin{figure}[H]
\centering
\includegraphics*[width=0.15\columnwidth]{tablou-de-bord-utilizator-meniu-universitate-administrator}
\includegraphics*[width=0.15\columnwidth]{tablou-de-bord-utilizator-meniu-universitate-profesor-student}
\caption{Tablou de bord Utilizator - Meniu Universitate Administrator(stânga), Profesor/Student(dreapta)}
\label{tablou-de-bord-utilizator-meniu-universitate}
\end{figure}

Butonul editare va deschide un formular de editare al universităţii, cel de părăsire va scoate utilizatorul din lista universităţii, iar cel de ştergere va şterge universitatea şi orice date asociate. În cazul butoanelor de părăsire şi ştergere, înainte să se execute comanda, utilizatorul va fi întrebat dacă este sigur ca operaţia este cea dorită.

\begin{figure}[H]
\centering
\includegraphics*[width=0.6\columnwidth]{tablou-de-bord-utilizator-formular-editare-universitate}
\caption{Tablou de bord Utilizator - Formular Editare Universitate}
\label{tablou-de-bord-utilizator-formular-editare-universitate}
\end{figure}

\begin{figure}[H]
\centering
\includegraphics*[width=0.45\columnwidth]{tablou-de-bord-utilizator-intrebare-siguranta-stergere}
\includegraphics*[width=0.45\columnwidth]{tablou-de-bord-utilizator-intrebare-siguranta-parasire}
\caption{Tablou de bord Utilizator - Întrebare siguranță ştergere(stânga), părăsire(dreapta)}
\label{tablou-de-bord-utilizator-intrebare-siguranta}
\end{figure}

\subsection{Structura Universităţii}

Odată selectată o universitate, utilizatorul va fi redirecţionat spre tabloul de bord al universităţii. Aici vor fi introduse noi funcţionalităţi în bara de instrumente, în partea de stânga vor apărea un buton cu detaliile universităţii, care va redirecţiona spre tabloul de bord al universităţii, şi o iconiţă de meniu care poate fi comutată, pentru a ţine meniul de navigare rapidă deschis sau pentru a-l închide. Tot aici, meniul utilizatorului va avea introduse butoane noi în funcţie de rolul pe care îl deţine în cadrul universităţii:
\begin{itemize}
	\item Student
		\begin{itemize}
			\item Note - vizualizarea rapidă a tuturor notelor grupate
		\end{itemize}
	\item Profesor
		\begin{itemize}
			\item Set Personal de Întrebări
		\end{itemize}
	\item Administrator
		\begin{itemize}
			\item Utilizatori - managementul utilizatorilor
		\end{itemize}
	\item Comun
		\begin{itemize}
			\item Activităţi Viitoare - Vizualizarea rapidă a următoarelor activităţi în ordine cronologică
		\end{itemize}
\end{itemize}

\begin{figure}[H]
\centering
\includegraphics*[width=\columnwidth]{tablou-de-bord-universitate-bara-de-instrumente}
\caption{Tablou de bord Universitate - Bară de Instrumente}
\label{tablou-de-bord-universitate-bara-de-instrumente}
\end{figure}

\begin{figure}[H]
\centering
\includegraphics*[width=0.4\columnwidth]{tablou-de-bord-universitate-meniu-navigare-rapida}
\caption{Tablou de bord Universitate - Meniu navigare rapidă}
\label{tablou-de-bord-universitate-meniu-navigare-rapida}
\end{figure}

\begin{figure}[H]
\centering
\includegraphics*[width=0.32\columnwidth]{tablou-de-bord-universitate-meniu-utilizator-administrator}
\includegraphics*[width=0.32\columnwidth]{tablou-de-bord-universitate-meniu-utilizator-profesor}
\includegraphics*[width=0.32\columnwidth]{tablou-de-bord-universitate-meniu-utilizator-student}
\caption{Tablou de bord Universitate - Meniu Utilizator Administrator(stânga), Profesor(mijloc), Student(dreapta)}
\label{tablou-de-bord-universitate-meniu-utilizator-administrator}
\end{figure}

Ca şi conţinut, va fi afişată lista de facultăţi din cadrul universităţii ca şi secţiune colapsabilă, în care va fi afişată lista de materii din cadrul respectivei facultăţi. Dacă utilizatorul deţine rolul de administrator în cadrul universităţii, vor apărea butoane de adăugare, editare sau ştergere a facultăţilor şi materiilor.

\begin{figure}[H]
\centering
\includegraphics*[width=\columnwidth]{tablou-de-bord-universitate-lista-facultati-administrator}
\caption{Tablou de bord Universitate - Listă facultăţi Administrator}
\label{tablou-de-bord-universitate-lista-facultati-administrator}
\end{figure}

\begin{figure}[H]
\centering
\includegraphics*[width=\columnwidth]{tablou-de-bord-universitate-lista-facultati-student}
\caption{Tablou de bord Universitate - Listă facultăţi Student}
\label{tablou-de-bord-universitate-lista-facultati-student}
\end{figure}

\begin{figure}[H]
\centering
\includegraphics*[width=0.45\columnwidth]{tablou-de-bord-universitate-formular-creare-facultate}
\includegraphics*[width=0.45\columnwidth]{tablou-de-bord-universitate-formular-editare-facultate}
\caption{Tablou de bord Universitate - Formular creare(stânga), editare(dreapta) Facultate}
\label{tablou-de-bord-universitate-formular-facultate}
\end{figure}

\begin{figure}[H]
\centering
\includegraphics*[width=0.45\columnwidth]{tablou-de-bord-universitate-formular-creare-materie}
\includegraphics*[width=0.45\columnwidth]{tablou-de-bord-universitate-formular-editare-materie}
\caption{Tablou de bord Universitate - Formular creare(stânga), editare(dreapta) Materie}
\label{tablou-de-bord-universitate-formular-materie}
\end{figure}


\begin{figure}[H]
\centering
\includegraphics*[width=0.15\columnwidth]{tablou-de-bord-universitate-meniu-facultate}
\caption{Tablou de bord Universitate - Meniu Facultate/Materie}
\label{tablou-de-bord-universitate-meniu-facultate}
\end{figure}

\subsection{Managementul utilizatorilor}

Odată apăsat butonul 'Utilizatori' din meniul dedicat administratorilor, utilizatorul va fi redirecţionat către tabloul de bord al tuturor utilizatorilor din cadrul universităţii. Aici se vor putea adăuga noi utilizatori, sau modifica sau şterge utilizatorii deja existenţi. Fiecare utilizator va avea un rol şi va fi înrolat la anumite facultăţi şi materii. Un caz mai special îl reprezintă utilizatorii cu rol de administrator, care nu vor fi înrolaţi la anumite facultăţi, aplicaţia îi va înrola automat la toate facultăţile şi materiile şi îi va ţine în această stare de fiecare dată când acestea se vor schimba. Totodată, administratorii nu vor avea posibilitatea ştergerii sau schimbării rolului altor administratori.

\begin{figure}[H]
\centering
\includegraphics*[width=\columnwidth]{tablou-de-bord-management-utilizatori}
\caption{Tablou de bord Management Utilizatori}
\label{tablou-de-bord-management-utilizatori}
\end{figure}

\begin{figure}[H]
\centering
\includegraphics*[width=0.7\columnwidth]{tablou-de-bord-management-utilizatori-formular-adaugare-utilizator}
\caption{Tablou de bord Management Utilizatori - Formular Adăugare Utilizator}
\label{tablou-de-bord-management-utilizatori-formular-adaugare-utilizator}
\end{figure}

\begin{figure}[H]
\centering
\includegraphics*[width=0.65\columnwidth]{tablou-de-bord-management-utilizatori-formular-editare-utilizator}
\caption{Tablou de bord Management Utilizatori - Formular Editare Utilizator}
\label{tablou-de-bord-management-utilizatori-formular-editare-utilizator}
\end{figure}

\begin{figure}[H]
\centering
\includegraphics*[width=0.3\columnwidth]{tablou-de-bord-management-utilizatori-alegere-rol-utilizator}
\caption{Tablou de bord Management Utilizatori - Alegere Rol Utilizator}
\label{tablou-de-bord-management-utilizatori-alegere-rol-utilizator}
\end{figure}

\begin{figure}[H]
\centering
\includegraphics*[width=0.75\columnwidth]{tablou-de-bord-management-utilizatori-formular-inrolare-la-facultati}
\caption{Tablou de bord Management Utilizatori - Formular Înrolare la Facultăţi}
\label{tablou-de-bord-management-utilizatori-formular-inrolare-la-facultati}
\end{figure}

\begin{figure}[H]
\centering
\includegraphics*[width=0.6\columnwidth]{tablou-de-bord-management-utilizatori-formular-inrolare-la-materii}
\caption{Tablou de bord Management Utilizatori - Formular Înrolare la Materii}
\label{tablou-de-bord-management-utilizatori-formular-inrolare-la-materii}
\end{figure}

\section{Funcţionalităţi profesor}

Odată selectată o materie, utilizatorul va fi redirecţionat spre tabloul de bord al materiei. Aici bara de instrumente îşi păstreaza aceleaşi funcţionalităţi, iar conţinutul va fi reprezentat de o listă de secţiuni, fiecare secţiune fiind colapsabilă şi conţinând activităţi. Profesorul va putea crea atât secţiuni, pentru o mai bună repartizare logică a activităţilor, cât şi activităţi. Crearea, editarea şi ştergerea secţiunilor au exact acelaşi format ca cel al facultăţilor.

\begin{figure}[H]
\centering
\includegraphics*[width=\columnwidth]{tablou-de-bord-materie-lista-sectiuni-profesor}
\caption{Tablou de bord Materie - Listă secţiuni Profesor}
\label{tablou-de-bord-materie-lista-sectiuni-profesor}
\end{figure}

\begin{figure}[H]
\centering
\includegraphics*[width=0.45\columnwidth]{tablou-de-bord-materie-formular-creare-sectiune}
\includegraphics*[width=0.45\columnwidth]{tablou-de-bord-materie-formular-editare-sectiune}
\caption{Tablou de bord Materie - Formular creare(stânga), editare(dreapta) Secţiune}
\label{tablou-de-bord-materie-formular-sectiune}
\end{figure}

\subsection{Crearea activităţilor}

Pentru crearea şi editarea activităţilor, formularul se va schimba în funcţie de tipul de activitate dorit.

Pentru tipurile materiale şi forum, acestea nu vor conţine nimic în plus faţă de formularul de bază al unei activităţi. Orice activitate are următoarele câmpuri de completat:
\begin{itemize}
	\item Nume - numele activităţii
	\item Descriere(opţional) - descriere sau textul activităţii
	\item Fişiere(opţional) - fişiere ajutătoare pentru parcurgerea activităţii de către student
\end{itemize}

\begin{figure}[H]
\centering
\includegraphics*[width=0.2\columnwidth]{tablou-de-bord-materie-meniu-alegere-tip-de-activitate}
\caption{Tablou de bord Materie - Meniu alegere tip de Activitate}
\label{tablou-de-bord-materie-meniu-alegere-tip-de-activitate}
\end{figure}

\begin{figure}[H]
\centering
\includegraphics*[width=0.45\columnwidth]{tablou-de-bord-materie-formular-creare-material}
\includegraphics*[width=0.45\columnwidth]{tablou-de-bord-materie-formular-editare-material}
\caption{Tablou de bord Materie - Formular creare(stânga), editare(dreapta) Material/Forum}
\label{tablou-de-bord-materie-formular-material}
\end{figure}

Pentru tipul temă, se vor mai adăuga 2 câmpuri:
\begin{itemize}
	\item Nota maximă - nota maximă pe care o poate obţine un student pentru realizarea temei
	\item Termen limită - data şi ora până când tema poate fi încărcată
\end{itemize}
Nota maximă trebuie să fie mai mare sau egal cu 0, iar termenul limită trebuie să reprezinte o dată mai mare decât data curentă.

\begin{figure}[H]
\centering
\includegraphics*[width=0.8\columnwidth]{tablou-de-bord-materie-campuri-tema}
\caption{Tablou de bord Materie - Câmpuri în plus pentru tipul Temă}
\label{tablou-de-bord-materie-campuri-tema}
\end{figure}

\begin{figure}[H]
\centering
\includegraphics*[width=0.32\columnwidth]{tablou-de-bord-materie-camp-alegere-data}
\includegraphics*[width=0.32\columnwidth]{tablou-de-bord-materie-camp-alegere-ora}
\includegraphics*[width=0.32\columnwidth]{tablou-de-bord-materie-camp-alegere-minut}
\caption{Tablou de bord Materie - Câmp alegere Timp - dată(stânga), oră(mijloc), minut(dreapta)}
\label{tablou-de-bord-materie-camp-alegere-timp}
\end{figure}

Pentru tipul test, se vor mai adăuga 6 câmpuri:
\begin{itemize}
	\item Timp deschidere - timp de deschidere al testului
	\item Timp închidere - timp de închidere al testului
	\item Timp limită(minute) - timpul limită maxim admis
	\item Amestecare întrebări - amestecă aleator ordinea întrebărilor
	\item Amestecare răspunsuri - amestecă aleator ordinea răspunsurilor fiecărei întrebări
	\item Întrebări - setul de întrebări pentru generarea testelor studenţilor, iar pentru fiecare întrebare, vom avea 2 câmpuri
		\begin{itemize}
			\item Ordinea - ordinea în lista de întrebări
			\item Nota maximă - nota maximă pe care o poate obţine un student pentru răspunsul corect la întrebare
		\end{itemize}
\end{itemize}
Timpul de deschidere şi cel de închidere trebuie să reprezinte o dată mai mare decât data curentă, timpul de închidere trebuie să fie după timpul de deschidere, timpul limită trebuie să fie mai mare decât 0, iar pentru fiecare întrebare în parte, nota maximă trebuie să fie mai mare sau egală cu 0.

\begin{figure}[H]
\centering
\includegraphics*[width=0.7\columnwidth]{tablou-de-bord-materie-campuri-test}
\caption{Tablou de bord Materie - Câmpuri în plus pentru tipul Test}
\label{tablou-de-bord-materie-campuri-test}
\end{figure}

\begin{figure}[H]
\centering
\includegraphics*[width=\columnwidth]{tablou-de-bord-materie-formular-adaugare-intrebari}
\caption{Tablou de bord Materie - Formular adăugare întrebări}
\label{tablou-de-bord-materie-formular-adaugare-intrebari}
\end{figure}

\subsection{Accesarea activităţilor}

Odată cu selectarea unei activităţi, utilizatorul va fi redirecţionat către un tablou de bord dedicat fiecărui tip de activitate. Orice activitate va avea afişat la început detaliile de bază: nume, descriere, fişiere, după care va continua cu detalii diferite în funcţie de tip.

\begin{figure}[H]
\centering
\includegraphics*[width=\columnwidth]{tablou-de-bord-activitate-detalii-de-baza}
\caption{Tablou de bord Activitate - Detalii de bază}
\label{tablou-de-bord-activitate-detalii-de-baza}
\end{figure}

În cazul temelor, se vor afişa cele 2 câmpuri în plus într-un tabel, Detalii Temă, iar sub tabel va fi lista cu studenţii participanţi, care au încărcat rezolvări la teme, împreuna cu detaliile lor: avatar-ul, punctajul, numele şi ultima dată de încărcare.

\begin{figure}[H]
\centering
\includegraphics*[width=\columnwidth]{tablou-de-bord-activitate-detalii-specifice-temelor}
\caption{Tablou de bord Activitate - Detalii specifice temelor}
\label{tablou-de-bord-activitate-detalii-specifice-temelor}
\end{figure}

Odată apăsat unul dintre studenţi, utilizatorul va fi redirecţionat către un tablou de bord dedicat notării studentului, unde vor fi afişate datele despre temă, rezolvarea încărcată de acesta şi un formular de notare în dreapta paginii.

\begin{figure}[H]
\centering
\includegraphics*[width=\columnwidth]{tablou-de-bord-notare-tema-student}
\caption{Tablou de bord - Notare temă Student}
\label{tablou-de-bord-notare-tema-student}
\end{figure}

În cazul testelor, se vor afişa într-un tabel detaliile specifice testului, iar sub acesta vor exista 2 secţiuni, prima este cea care conţine detalii despre întrebările care aparţin de test, aceasta fiind o secţiune colapsabilă, iar cea de-a doua este lista cu studenţii care au participat la test.

\begin{figure}[H]
\centering
\includegraphics*[width=\columnwidth]{tablou-de-bord-activitate-detalii-specifice-testelor}
\caption{Tablou de bord Activitate - Detalii specifice testelor}
\label{tablou-de-bord-activitate-detalii-specifice-testelor}
\end{figure}

\begin{figure}[H]
\centering
\includegraphics*[width=\columnwidth]{tablou-de-bord-activitate-intrebari-test}
\caption{Tablou de bord Activitate - Întrebări test}
\label{tablou-de-bord-activitate-intrebari-test}
\end{figure}

\begin{figure}[H]
\centering
\includegraphics*[width=\columnwidth]{tablou-de-bord-activitate-lista-studenti-test}
\caption{Tablou de bord Activitate - Listă studenţi Test}
\label{tablou-de-bord-activitate-lista-studenti-test}
\end{figure}

Odată apăsat unul dintre studenţi, utilizatorul va fi redirecţionat către un tablou de bord dedicat notării studentului, dar deoarece notarea testelor se face automat, această pagină este strict pentru vizionarea testului. Ca şi conţinut se vor afişa întrebările şi răspunsurile la acestea, respectiv nota obţinuta la fiecare întrebare, iar în dreapta se va regăsi o metodă de navigare rapidă prin întrebările testului.

\begin{figure}[H]
\centering
\includegraphics*[width=\columnwidth]{tablou-de-bord-notare-test-student}
\caption{Tablou de bord - Notare test Student}
\label{tablou-de-bord-notare-test-student}
\end{figure}

În cazul forumurilor, se vor afişa comentariile adăugate de participanti, orice rol este permis.

\begin{figure}[H]
\centering
\includegraphics*[width=\columnwidth]{tablou-de-bord-forum}
\caption{Tablou de bord Activitate - Forum}
\label{tablou-de-bord-forum}
\end{figure}

\begin{figure}[H]
\centering
\includegraphics*[width=0.5\columnwidth]{tablou-de-bord-activitate-adaugare-comentariu-forum}
\caption{Tablou de bord Activitate - Adăugare comentariu Forum}
\label{tablou-de-bord-activitate-adaugare-comentariu-forum}
\end{figure}

\subsection{Crearea setului personal de întrebări}

La selectarea Setului personal de întrebări din meniu, utilizatorul va fi redirecţionat către tabloul de bord cu întrebări. Acesta va conţine o listă de categorii de întrebări, pentru o mai bună repartizare logică a acestora, iar fiecare categorie va fi o secţiune colapsabilă care va conţine întrebările specifice.

\begin{figure}[H]
\centering
\includegraphics*[width=\columnwidth]{tablou-de-bord-set-intrebari}
\caption{Tablou de bord Set Întrebări}
\label{tablou-de-bord-set-intrebari}
\end{figure}

\begin{figure}[H]
\centering
\includegraphics*[width=0.45\columnwidth]{tablou-de-bord-set-intrebari-creare-categorie}
\includegraphics*[width=0.45\columnwidth]{tablou-de-bord-set-intrebari-editare-categorie}
\caption{Tablou de bord Set Întrebări - Formular creare(stânga), editare(dreapta) Categorie}
\label{tablou-de-bord-set-intrebari-categorie}
\end{figure}

Formularul de creare a unei întrebări cuprinde următoarele câmpuri:
\begin{itemize}
	\item Tip de activitate - care poate fi cu răspuns unic, sau cu răspuns multiplu
	\item Nume - numele întrebării
	\item Text - textul întrebării
	\item Răspunsuri - răspunsurile întrebării, iar pentru fiecare răspuns trebuie completate 2 câmpuri
		\begin{itemize}
			\item Ordine - ordinea din lista de răspunsuri
			\item Text - textul răspunsului
			\item Procentaj/Fracţiune - ponderea pe care o are răspunsul respectiv pentru întrebare, poate lua valori între -100 şi 100
		\end{itemize}
\end{itemize}

\begin{figure}[H]
\centering
\includegraphics*[width=0.7\columnwidth]{tablou-de-bord-set-intrebari-formular-intrebare}
\caption{Tablou de bord Set Întrebări - Formular Întrebare}
\label{tablou-de-bord-set-intrebari-formular-intrebare}
\end{figure}

Pentru o calculare automată corectă a notelor de la teste, exista câteva indicaţii care trebuie respectate.
\begin{itemize}
	\item Procentajul poate lua valori între -100 şi 100
	\item Pentru întrebările cu răspuns unic, doar 1 singur răspuns ar trebui să aibă procentajul = 100, restul ar trebui să fie = 0
	\item Pentru întrebările cu răspuns multiplu, suma tuturor procentajelor pozitive ar trebui să fie = 100
	\item Pentru întrebările cu răspuns multiplu, suma tuturor procentajelor negative ar trebui să fie = -100
\end{itemize}

\begin{figure}[H]
\centering
\includegraphics*[width=0.7\columnwidth]{tablou-de-bord-set-intrebari-indicatii-notare-automata}
\caption{Tablou de bord Set Întrebări - Indicaţii Notare Automată}
\label{tablou-de-bord-set-intrebari-indicatii-notare-automata}
\end{figure}

\section{Funcţionalităţi student}

Ajungând în tabloul de bord al materiei, utilizatorii cu rol de student vor vedea aceleaşi informaţii ca şi cei cu rol de profesor, în afară de butoanele de creare, editare sau ştergere ale secţiunilor şi activităţilor.

\begin{figure}[H]
\centering
\includegraphics*[width=0.7\columnwidth]{tablou-de-bord-materie-lista-sectiuni-student}
\caption{Tablou de bord Materie - Listă secţiuni Student}
\label{tablou-de-bord-materie-lista-sectiuni-student}
\end{figure}

\subsection{Participarea la activităţi}

Odată selectată o activitate, utilizatorul va fi redirecţionat către tabloul de bord al activităţii, care va afişa ca şi în cazul utilizatorilor cu rol de profesor, detalii de bază despre activitate, dar şi detalii specifice studentului. Diferenţele notabile sunt pentru tipurile temă şi test.

În cazul temelor, studentului i se vor afişa detalii personale legate de tema respectivă, notă, ultima dată de încărcare, dacă timpul limită nu s-a scurs şi o modalitate de încărcare a temei.

\begin{figure}[H]
\centering
\includegraphics*[width=\columnwidth]{tablou-de-bord-activitate-incarcare-tema}
\caption{Tablou de bord Activitate - Încărcare temă}
\label{tablou-de-bord-activitate-incarcare-tema}
\end{figure}

În cazul testelor, ca şi în cazul temelor, studentului i se vor afişa detalii personale legate de testul respectiv, notă, timp deschidere test, timp trimitere test pentru corectare.

\begin{figure}[H]
\centering
\includegraphics*[width=\columnwidth]{tablou-de-bord-activitate-detalii-test-student}
\caption{Tablou de bord Activitate - Detalii test Student}
\label{tablou-de-bord-activitate-detalii-test-student}
\end{figure}

În funcţie de timpul curent şi detaliile testului, utilizatorului i se va afişa una dintre cele 5 stări ale butonului de interacţionare:
\begin{itemize}
	\item Nimic - testul nu este activ încă sau testul a expirat, utilizatorul nu şi-a început încercarea
	\item Începe - testul este activ, utilizatorul nu şi-a început încercarea
	\item Continuă - testul este activ, utilizatorul şi-a început încercarea
	\item Verifică(blocat) - testul este activ, utilizatorul a trimis rezolvarea pentru notare
	\item Verifică - testul a expirat, utilizatorul şi-a început încercarea
\end{itemize}

Butoanele Începe şi Continuă vor redirecţiona studentul spre încercarea curentă a testului. Starea testului, răspunsurile la întrebări, sunt salvate incremental, în timp ce studentul rezolva testul, în cazul unei urgenţe sau deconectări, studentul nu va pierde tot progresul pe care l-a făcut.

Tabloul de bord al testului activ este format din meniul pentru navigare rapidă în dreapta paginii, iar ca şi conţinut se vor afişa întrebările, una câte una, detalii despre aceasta, şi timpul rămas din încercare. Toate celelalte butoane şi meniuri din bara de instrumente vor fi ascunse, pentru cât mai puţine distrageri şi/sau apăsări din greşeală.

\begin{figure}[H]
\centering
\includegraphics*[width=\columnwidth]{tablou-de-bord-test-activ}
\caption{Tablou de bord Test activ}
\label{tablou-de-bord-test-activ}
\end{figure}

\begin{figure}[H]
\centering
\includegraphics*[width=0.7\columnwidth]{tablou-de-bord-test-activ-intrebare-raspuns-unic}
\caption{Tablou de bord Test activ - Întrebare răspuns unic}
\label{tablou-de-bord-test-activ-intrebare-raspuns-unic}
\end{figure}

\begin{figure}[H]
\centering
\includegraphics*[width=0.7\columnwidth]{tablou-de-bord-test-activ-intrebare-raspuns-multiplu}
\caption{Tablou de bord Test activ - Întrebare răspuns multiplu}
\label{tablou-de-bord-test-activ-intrebare-raspuns-multiplu}
\end{figure}

După ce testul a expirat, utilizatorul îşi poate verifica răspunsurile la întrebări şi notele.

\begin{figure}[H]
\centering
\includegraphics*[width=\columnwidth]{tablou-de-bord-verificare-test}
\caption{Tablou de bord Verificare Test}
\label{tablou-de-bord-verificare-test}
\end{figure}

% \subsection{Vizualizarea notelor}
% TODO: do this when you have time
% \subsection{Vizualizarea urmatoarelor activitati}
% TODO: do this when you have time

% \section{Baza de date}
% TODO: do this when you have time - logical separation

% \section{Securitate}
% TODO: do this when you have time - jwt, roles, scopes

\chapter*{Concluzii}

Proiectul implementează funcţionalităţile cele mai importante pentru desfăşurarea activităţii de predare/testare în mediu online. Această metodă nu are rolul de a înlocui desfăşurarea activităţii în mediu fizic, ci de a o îmbunătăţii. Consider ca integrarea unei platforme de e-learning în cadrul oricărei universităţi vine cu beneficii atât pentru studenţi, cât şi pentru profesori. Trecerea în mediu online a unor tipuri de activităţi poate salva timp, efort, chiar şi bani pentru laturile implicate.

\addcontentsline{toc}{chapter}{Concluzii}

\bibliographystyle{unsrt}
\bibliography{referinte}

% \appendix

\end{document}

